% Version 0; preprint format; Written by SH

\documentclass[preprint]{aastex}

% Packages
\usepackage{emulateapj5}
\usepackage{apjfonts}
\usepackage{amssymb, amsmath}
\usepackage{graphicx}
\usepackage{CJK}
\usepackage[usenames]{color}

% Figure extention
\DeclareGraphicsExtensions{.pdf,.png,.jpg}
\bibliographystyle{apj}
\input psfig.tex

%%%%%%%%%%%%: User Defined Commands %%%%%%%%%%%%

% Song Huang's definition 
\def\arcsec{{\prime\prime}}
\def\arcmin{{\prime}}
\def\degree{{\circ}}
\def\h{\hskip -3 mm}
\def\aa{{A\&A}}
\def\aas{{ A\&AS}}
\def\aj{{AJ}}
\def\al{$\alpha$}
\def\bet{$\beta$}
\def\amin{$^\prime$}
\def\annrev{{ARA\&A}}
\def\apj{{ApJ}}
\def\apjs{{ApJS}}
\def\asec{$^{\prime\prime}$}
\def\deg{$^{\circ}$}
\def\ddeg{{\rlap.}$^{\circ}$}
\def\dsec{{\rlap.}$^{\prime\prime}$}
\def\cc{cm$^{-3}$}
\def\etal{{\ et al.~}}
\def\flamb{erg s$^{-1}$ cm$^{-2}$ \AA$^{-1}$}
\def\flux{erg s$^{-1}$ cm$^{-2}$}
\def\fnu{erg s$^{-1}$ cm$^{-2}$ Hz$^{-1}$}
\def\hst{{\it HST}}
\def\kms{km s$^{-1}$}
\def\lamb{$\lambda$}
\def\lax{{$\mathrel{\hbox{\rlap{\hbox{\lower4pt\hbox{$\sim$}}}\hbox{$<$}}}$}}
\def\gax{{$\mathrel{\hbox{\rlap{\hbox{\lower4pt\hbox{$\sim$}}}\hbox{$>$}}}$}}
\def\simlt{\lower.5ex\hbox{$\; \buildrel < \over \sim \;$}}
\def\simgt{\lower.5ex\hbox{$\; \buildrel > \over \sim \;$}}
\def\micron{{$\mu$m}}
\def\mnras{{MNRAS}}
\def\nat{{Nature}}
\def\pasp{{PASP}}
\def\perang{\AA$^{-1}$}
\def\peryr{yr$^{-1}$}
\def\pp{\parshape 2 0truein 6.1truein .3truein 5.5truein}
\def\reference{\noindent\pp}
\def\refindent{\par\noindent\parskip=2pt\hangindent=3pc\hangafter=1 }
\def\sb{mag~arcsec$^{-2}$}
\def\solum{$L_\odot$}
\def\solmass{$M_\odot$}
\def\sigs{$\sigma_*$}
\newcommand{\lt}{<}
\newcommand{\gt}{>}
\def\galfit{{\tt GALFIT}}
\def\ser{{S\'{e}rsic\ }}
\def\redm{\texttt{redMaPPer\ }}
\def\redbcg{$\Lambda \ge 30$}
\def\nonbcg{$\Lambda < 30$}
\def\logms{$\log (M_{\star}/M_{\odot})$}
\def\logmh{$\log (M_{\mathrm{Halo}}/M_{\odot})$}

% Commenting:
\newcommand{\todo}[1]{\textcolor{red}{\textbf{TODO: #1}}}
\newcommand{\plan}[1]{\textcolor{cyan}{#1}}
\newcommand{\addref}{{\textcolor{red}{REF}}}
\newcommand{\comment}[2]{\textcolor{blue}{\textbf{[Comment (#1): #2]}}}
\newcommand{\song}[1]{\textcolor{magenta}{\textbf{[Song: #1]}}}
\newcommand{\alexie}[1]{\textcolor{blue}{\textbf{[Alexie: #1]}}}
\newcommand{\kevin}[1]{\textcolor{green}{\textbf{[Kevin: #1]}}}

%%%%%%%%%%%%: Header and Version %%%%%%%%%%%%

\slugcomment{Draft version 0}
\email{song.huang@ipmu.jp}
\shorttitle{ENVIRONMENT DEPENDENCE OF MASS-SIZE RELATION}
\shortauthors{HUANG ET AL.}

\begin{document}

\begin{CJK*}{UTF8}{gbsn}

%%%%%%%%%%%%: Title and Affiliations %%%%%%%%%%%%

\title{The Environmental Dependence of Structures for Massive Galaxies
from the Hyper Suprime-Cam Survey}

\author{Song Huang (黄崧)\altaffilmark{1} 
        Alexie Leauthaud\altaffilmark{1}, 
        Kevin Bundy\altaffilmark{1},
        Michael Strauss\altaffilmark{2},
        Yen-Ting Lin\altaffilmark{3},
        Rachel Mandelbaum\altaffilmark{4}
        }
\date{}                                          

\altaffiltext{1}{Kavli Institute for the Physics and Mathematics of the 
    Universe, The University of Tokyo Institutes for Advanced Study, 
    the University of Tokyo (Kavli IPMU, WPI), Kashiwa 277-8583, Japan}
\altaffiltext{2}{Princeton University Observatory, 
    Peyton Hall, Princeton, NJ 08544, USA}
\altaffiltext{3}{Institute of Astronomy and Astrophysics, Academia Sinica, 
    P.O.~Box 23-141, Taipei 10617, Taiwan}
\altaffiltext{4}{Carnegie-Mellon University \todo{correct affiliation here}}

%%%%%%%%%%%%: Abstract and Keywords %%%%%%%%%%%%

\begin{abstract}
    \todo{Place holder; This is the abstract used for giving a talk.}\\
    Under the most popular formation scenario, the structures of massive central galaxies
    depend on environment at fixed stellar mass due to different assembly histories shaped
    by their host halos. Yet, clear evidence of such effect is still lacking. Using deep,
    multi-band images for a large sample of massive galaxies at 0.2 < z < 0.5 from the
    Hyper Suprime-Cam (HSC) survey, we discover subtle, but systematic structural
    difference for massive galaxies in low and high mass haloes. The differences are
    consistent with richer merger history in more massive halos. We show that the average
    profiles of mass, shape, and color, along with relations among masses within different
    radius (as proxy of mass assembled at different time) can help us gain more insights
    of their assembly history, and the weak lensing analysis enabled by HSC survey further
    helps us connect the differences we find to the average halo properties.
\end{abstract}
\keywords{galaxies: elliptical and lenticular, cD --- galaxies: formation --- 
          galaxies: photometry --- galaxies: structure --- galaxies: surveys}

\maketitle

%%%%%%%%%%%%: Main Text %%%%%%%%%%%%

%% ------------------------------------------------------------------------------------ %% 
\section{Introduction}

    \plan{Scientific Background}
    \begin{itemize}
        \item \plan{Massive galaxies are important cosmic probe and unique labs to study 
            galaxy evolution.} 
        \item \plan{Briefly explain why massive early-type galaxies are important through 
            the difficulties of stellar-halo mass relation and stellar mass function at the 
            high-mass end.}
        \item \plan{Brief summary of the current understanding of their cosmic assembly 
            history.}
        \item \plan{Explain why we should care about the environment, and why we expect to 
            see some environemtnal dependence in the structure and other properties of
            massive galaxies.}
        \item \plan{Brief review of the current observations.  It is still not clear 
            whether there is a clear environmental dependence.}
    \end{itemize}

    \plan{Observational difficulties (a.k.a Why we need HSC)}
    \begin{itemize}
        \item \plan{Explain why it is important to study the mass distribution of 
            massive galaxies out to large physical radius; and, given their unique light 
            profiles, why it is more difficult to study them compared to late-type 
            galaxies.}
        \item \plan{Very brief summary of past observational efforts, and why they are not 
            good enough (Not enough number of really massive galaxy; Shallow images; 
            Background subtraction issue; and stacking analysis can be dangerous as while)}
    \end{itemize}
    
    \plan{Basica idea of this work}
    \begin{itemize}
        \item \plan{Here we take advantages of the ambitious Hyper-Suprime camera 
            survey....}
    \end{itemize}

    The paper is organized as follows. Section~2 gives a brief overview of the HSC
    observation and data reduction.  We will also summarize the process of sample
    selection.  In Section~3, we will describe the method for deriving the stellar mass 
    surface density profile.  The main results are summarized in Section~4.  Section~5 
    provides discussions of the assumptions used in this work, the potentially interesting
    physical implications, and several future improvements, ending with a summary in
    Section~6.

    All the magnitudes used here are in AB system (\addref).  
    Within this work, we assume $H_0$ = 70~km~s$^{-1}$ Mpc$^{-1}$, ${\Omega}_m=0.3$, 
    and ${\Omega}_{\Lambda}=0.7$.
    
%% ------------------------------------------------------------------------------------ %% 

\section{Observations and Sample Selection}

\subsection{The Hyper Suprime-Cam Survey}

    The Subaru Strategic Program (SSP, \addref ) makes use of the new prime-focus 
    camera, the Hyper Suprime-Cam (HSC; \addref: Miyazaki\etal 2012), on the 8.2-m 
    Subaru telescope at Mauna Kea.  
    Taking advantage of the large field of view (FoV; 1.5 deg in diameter) of HSC, 
    this ambitious multi-layer photometric survey will cover $\sim 1400$ deg$^2$ of 
    sky in 5 broad bands (\textit{g r i z Y}) to the depth of $r \sim 26$ mag in the 
    \texttt{WIDE} part in the next few years.  
    This work is based on the internal data release \texttt{S15B}, which covers 
    $\sim 100$ square of degree of sky in all 5-band to the required depth of 
    \texttt{WIDE} survey.  
    The regions covered by this release are overlapped with several previous 
    spectroscopic survey (e.g. SDSS/BOSS \addref; GAMA \addref).

    The data are processed with \texttt{hscPipe 4.0.1}, a derivative of the Large 
    Synoptic Survey Telescope (LSST) pipeline (\addref: Ivezic\etal 2008,
    Axelrod\etal 2010), modified for use with Suprime-Cam and Hyper Suprime-Cam.  
    \texttt{hscPipe} first bias subtract, flat field, model background, and 
    perform object detection and measurement on the single exposure data. 
    Then, different exposures are warped onto a common World Coordinate System 
    (WCS) and combined into final images with improved signal-to-noise ratio 
    (SNR) after astrometric and photometric calibration.  
    The pixel scale of the combined images is 0.168$\arcsec$.
    The photometric calibration is based on data obtained from the Panoramic 
    Survey Telescope and Rapid Response System (Pan-STARRS) 1 imaging survey 
    (\addref: Schlafly\etal 2012, Tonry\etal 2012, Magnier\etal 2013). 
    To achieve consistent deblending and photometry across all bands, the 
    \texttt{hscPipe} will perform multi-band post-processing on the combined 
    images.  
    The footprints and peaks of detected sources on each band will be 
    merged into a single catalog.    
    This consistent set of peaks and footprints is used as starting point 
    for deblend and measure objects on the combined images of each band.  
    These measurements are then merged into a reference catalog. 
    After fixing the centroids, shape, and other non-amplitude parameters 
    of every object in this catalog, \texttt{hscPipe} will derive forced 
    photometry measurements at each band. 
    Details of \texttt{hscPipe} and the multi-band processing method will 
    be presented in \addref (Bosch\etal 2017?).     
          
    In Figure 1, we compare the false color ({\tt gri}) images of three nearby 
    massive ETGs from both SDSS and HSC surveys.  
    The images are generated using similar scaling and color schemes, and clearly 
    demonstrate HSC's capability to reach to low surface brightness domain.  
    In $i$-band (ignoring the slight difference in filter response curve), the 
    HSC \texttt{WIDE} image is $2.5$-$3.0$ magnitude deeper than SDSS on average.  
    This gives us huge advantage in exploring the outskirt of massive ETGs given 
    their extended, shallow outer surface brightness profiles. 
    
    Motivated by the requirement of weak lensing analysis, the $i$-band data 
    typically has the best seeing in all five bands (the median seeing is around 
    FWHM$\sim 0.8\arcsec$).  Therefore, we will mostly use the $i$-band images
    to study the structure of galaxies. 
    
\subsection{Massive central galaxies in different environments}

    The main scientific goal of this work is to investigate mass assembly history 
    of massive, central ETGs, and its dependence on environments (or host dark halo 
    mass) through their the spatial distribution of stellar mass.  
    
    To achieve this goal, we will select \logms$ > 11.5$ massive central 
    galaxies within host halo mass larger and smaller than \logmh$= 14.0$
    at $0.2 < z < 0.5$.  
    As both $M_{\star}$-$M_{\mathrm{Halo}}$ relation and stellar mass function 
    are still quite uncertain at \logms$ > 11.5$, it is of great interest to 
    investigate the structure of galaxies in this region carefully.
    And, this stellar mass cut can greatly reduce the contamination of 
    satellites (e.g. \addref: van~Uitert\etal 2016).  
    Meanwhile, under the adopted cosmology, $1\arcsec$ equals 3.3 and 6.1 kpc 
    at redshit 0.2 and 0.5.  
    Hence, our redshift bin enables us to reliable measure the total stellar 
    mass within the inner 5 to 10 kpc (where the ``in-situ'' component should 
    still dominates) of massive galaxies at the high redshift end assuming the
    typical seeing of data. 
    At the same time, the depth of the data still allows to study the very 
    outskirt ($\sim 100$ kpc) of these galaxies at $z\sim 0.5$.  
    Also, such limits on mass and redshift ensure us a sizable sample while let 
    us safely ignore any significant structural evolution (no star formation, 
    lower merger rate \etal. e.g. \addref). 
        
    Based on recent constraints of $M_{\star}$-$M_{\mathrm{Halo}}$ 
    relation (e.g. \addref), above \logms$ = 11.5$, there is a large scatter 
    of halo mass at fixed stellar mass.  
    At \logmh$<14.0$, these massive galaxies are mostly the centrals of small 
    groups; while at \logmh$>14.0$, they start to become the centrals of very 
    massive groups and galaxy clusters.   
    Limited by the sample size and the capability of measureing halo mass for 
    individual central galaxy, we will simply separate the sample into two 
    broad halo mass bins with the help of the \redm cluster catalog 
    \footnote{See: http://risa.stanford.edu/redmapper/} (
    \texttt{v5.10}, e.g. \addref: Rykoff\etal 2014; Rozo\etal 2015b).
    These clusters are selected from the SDSS DR8 photometric data 
    using overdensity of red-sequence galaxies. 
    For each cluster, the catalog provides robust estimations of 
    photometric redshift $z_{\Lambda}$ and richness $\Lambda$, along with 
    the best candidate of the central galaxy (the one with the highest 
    central probability \texttt{P\_CEN}).
    Information about the identified member candidates is also provided 
    separately. 
    Please see Rozo\etal (2014, 2015a, 2015b) for more details about the 
    performance of the \redm cluster catalog.      
    Several works (e.g. Li\etal 2015; Saro\etal 2015; Farahi\etal 2016; 
    Simet\etal 2016) have tried to calibrate the $M_{200c}-\Lambda$ relation 
    using different methods.
    Despite the slightly different calibrations derived, it is safe to 
    assume that most clusters identified by \redm ($\Lambda > 20$) have
    $\log (M_{200c}/M_{\odot}) \geq 14.0$.  
    Considering the typical uncertainty of richness, and the fact that 
    \redm catalog starts to become incomplete toward low richness 
    ($\Lambda < 40$) end at $z > 0.33$, we will focus on the $\Lambda > 30$
    clusters in this work.  
    Based on the calibrations from Farahi\etal (2016) or Simet\etal (2016), 
    such richness cut gives us haloes with 
    $M_{200c} \geq 1.56\pm0.35 \times 10^{14} M_{\odot}$ or 
    $M_{200c} \geq 1.60\pm0.11 \times 10^{14} M_{\odot}$.  
    Therefore, \redm can provide us a sample of massive central galaxies in
    \logmh$\geq 14.0$ haloes.  
    And, we assume that, in the same footprints and redshift range, the 
    \logms$\geq 11.5$ galaxies outside \redm clusters are most likely to be 
    the central of \logmh$< 14.0$ haloes. 
    
    According to Leauthaud\etal (2016), all $0.2 < z < 0.5$ and \logms$\geq 11.5$
    galaxies should have $i_{\mathrm{SDSS, cModel}} \leq 21.0$ mag. 
    Ignoring the tiny difference in response curves between SDSS and HSC 
    $i$-band filters, we started by selecting all galaxies with 
    $i_{\mathrm{HSC, cModel}} \leq 21.5$ in the regions that are covered 
    in all five filters, and have already reached the expected depth of 
    \texttt{WIDE} survey.  
    The details of the HSC \texttt{cModel} photometry, and its performence
    will be described in Bosch\etal in. prep. 
    It is quite similar to the SDSS \texttt{cModel} in principle as it also 
    fits the total flux of an object using a combination of de~Vaucouleur 
    and exponential components after the PSF convolution is considered.  
    In our selection, we require well defined centroids and \texttt{cModel} 
    magnitude in all five bands. 
    \song{Should we discuss the impact of failed cModel photometry?}
    A series of quality control cuts are applied to remove objects that are 
    affected by saturation, cosmic ray, other optical artefacts.  
    After these cuts, a total number of \plan{N1} galaxies left in the 
    sample () \todo{Finish this part}  
    (This sample will be referred as \texttt{phoAll}). 
    
    We match this sample with the central galaxies of \redm catalogs using 
    a $2.0\arcsec$ radius, and it results in \plan{N2} galaxies.  
    A small fraction of \redm centrals in the our footprints do not have 
    matched object in our sample due to severe contamination from optical 
    artefacts or bleeding trails from saturated objects.  
    And slight change of the mathcing radius has no impact of this sample.  
    For galaxy in this sample (will be referred as \texttt{redBCG}), 
    the photometric redshift of the cluster is available for each of them; 
    while for \plan of them, the spectroscopic redshift from SDSS DR8 is 
    also provided by \redm catalog when available.  
    
    For the vast majority of galaxies that are not the centrals of very massive
    haloes, we need other sources for their redshfit estimations. 
    Although there has been many efforts for deriving photometric redshift 
    (photo-$z$) using HSC five-band photometry, it is still a working progress 
    now, and its performence at $z \leq 0.5$ is expected to be less satisfying 
    due to the lack of information at the shorter wavelength side of the 
    4000$\AA$ break.  
    Therefore, we will only rely on external spectroscopic redshifts (spec-$z$)
    for this work.  
    The HSC database compiles a catalog of external spec-$z$ for this data 
    release by matching the detected objects with public data of several 
    spectroscopic surveys (e.g. SDSS/BOSS \addref; GAMA \addref; PRIMUS \addref;
    \etal). 
    Duplicated matches from different sources are merged through interal 
    matching using $0.5\arcsec$ radius.  
    
    We match the \texttt{phoAll} sample with this external spec-$z$ catalog 
    using a $1.0 \arcsec$ radius, and it leads to \plan{N3} results. 
    There are \plan{N4} objects at $0.2 \leq z \leq 0.5$.   
    The \texttt{redBCG} sample is also matched in the same way. 
    For ...
    \todo{Finish it...}
    
    
    
    


   
    \plan{Selection of the \texttt{PHOTO} sample}
    \begin{itemize}
        \item \plan{\texttt{PRIMARY} detection: In the inner region of the \texttt{Tract}
            and \textt{Patch}; has no \texttt{CHILD}.}
        \item \plan{Require the galaxy to be detected in all 5-band}
        \item \plan{For $i$-band detection, we require the object is covered by at least 
            4 \texttt{Visit}.}
        \item \plan{Has useful \texttt{cModel} photometry in all 5-band}
        \item \plan{$i$-band \texttt{cModel} magnitude brigther than 21.0 mag (justify this 
            later}
        \item \plan{Has to be extended object.}
        \item \plan{Has reliable central position and shape using SDSS algorithm}
        \item \plan{Typical quality cuts in $griz$-band.}
    \end{itemize}
    
    \plan{Selection of the \texttt{SPECZ} sample}
    \begin{itemize}
        \item \plan{Brief summary of the HSC spec-z catalog.}
        \item \plan{Internal catalog match results in 121998 galaxies}
        \item \plan{Brief break down of the spec-z sample; Most of the spectroscopic 
            redshift come from SDSS/BOSS and GAMA survey}
        \item \plan{Only keep objects at $z < 0.6$} \todo{Justify this later}
    \end{itemize}

\subsection{Central galaxies of $\Lambda > 20$ haloes using the \redm clusters}

    To select the central galaxies of cluster-level dark matter haloes, we make use of the 
    \redm cluster catalog \footnote{See: http://risa.stanford.edu/redmapper/}. 
    
    \plan{Match to the \redm catalog}
    \begin{itemize}
        \item \plan{For this work, we use the \redm catalog based on SDSS DR8 
            photometry.} \todo{Need some explanation}
        \item \plan{Match with the \texttt{PHOTO} sample using a 1.0$\arcsec$
            matching radius, results in 704 clusters. (Justify this matching radius)}
        \item \plan{Select the \redm clusters at $z < 0.6$.}
        \item \plan{There is a small population of central galaxies from \redm that
            are within the HSC footprint, but does not have matched photometry, 
            briefly explain them}
    \end{itemize}

\subsection{Galaxies in $\Lambda < 20$ haloes}

    \begin{itemize}
        \item \plan{Remove the galaxy that is close to any \redm cluster}
        \item \plan{Remove the galaxy that is close to any bright star} 
            \todo{Justify why we do not do this to the \redm central sample}
    \end{itemize}
    
%% ------------------------------------------------------------------------------------ %% 

\section{Data Reduction}

\subsection{1-D surface brightness profile}

    \plan{Basica ideas}
    \begin{itemize}
        \item \plan{Briefly explain that 1-D surface brightness profile is a very old 
            fashion, but robust and straightforward way to describe the light or mass 
            distribution of a galaxy.  Although it is not exactly ``model-independent'', 
            it indeed has the advantages of not easily affected by complex substrctures 
            within the galaxy comparing with the more popular 2-D modelling method. }
        \item \plan{Due to the uniquely extended nature of the light distributions of 
            these massive ETGs, it is still unclear which is the most appropriate 2-D 
            model for them (Can not be well described by 1-\ser model). }
    \end{itemize}
    
    \plan{Step by step description}
    \begin{itemize}
        \item \plan{Generation of cutout image in multiple bands}: \\
            \begin{enumerate}
                \item \plan{Choice of image size: make sure that the cutout covers out 
                    to at least 500 kpc from the center of the galaxy.}
                \item \plan{Also generate the bad pixel masks, and the reconstructed 
                    PSF model for the galaxy center.}
            \end{enumerate}
        \item \plan{Generation of object masks}: \\
            \begin{enumerate}
                \item \plan{Briefly explain why we need to our own photometry instead 
                    of just using the cModel results from the HSC pipeline.}
                \item \plan{We need two kinds of masks: mask of all objects that will 
                    be used to measure the sky value; mask of objects expect for the 
                    central galaxy we want to derive surface brightness profile.  
                    Briefly describe their requirements.}
                \item \plan{Object detection using \texttt{SEP} Python library.}
            \end{enumerate}
        \item \textbf{Re-measurement and correction of sky background}:
    \end{itemize}
    
\subsection{Average mass-to-light ratio}

    \begin{itemize}
        \item \plan{Briefly explain the method to convert surface brightness profiles 
            into mass density profile, focusing on the key assumptions.}
        \item \plan{Describe the SED fitting process.}    
        \item \plan{Briefly discuss the impact of choices of SSP model, IMF, priors of 
            parameters for SED fitting}
    \end{itemize}

\subsection{Basic properties of the sample}

    \begin{itemize}
        \item \plan{Compare stellar mass from integration of 1-D profile and the 
            \texttt{cModel} photometry}
        \item \plan{Stellar mass, redshift, K-corrected color distributions.}
        \item \plan{Red sequence properties}    
        \item \plan{Potential ``contaminations'' from disc galaxies}
    \end{itemize}
    
    \todo{Justify why we can assume most very massive galaxies in \nonbcg haloes are
        central galaxies}

%% ------------------------------------------------------------------------------------ %% 

\section{Results}

\subsection{Impact of deep photometry on the luminosity and stellar mass function}

\subsection{Comparison of stellar mass density profiles}

    \plan{Brief summary of the process of sample matching}

\subsubsection{Using stellar mass derived from the GAMA survey}

\subsubsection{Using stellar mass within 10 kpc aperture}

\subsubsection{Using stellar mass within 100 kpc aperture}

\subsection{Relations between stellar mass within different physical apertures}

\subsection{$M_{\ast}-R_{\mathrm{50}}$ relation}

%% ------------------------------------------------------------------------------------ %% 

\section{Discussion}

\subsection{Implications on assembly history of massive galaxies}

    \plan{Comparison of mass profiles with previous studies at high redsfhit}

\subsection{Ellipticity profiles}

\subsection{The impact of $M/L$ gradient}

\subsection{Massive satellites of the $\Lambda > 20$ haloes}
\song{TBD: optional}

\subsection{Connection and difference with the Intra-Cluster Light}

\subsection{Future improvements}

    \begin{enumerate}
        \item \plan{Improvements of HSC data reduction: better sky modelling; 
            \texttt{redMaPPer} clusters using HSC photometry.}
        \item \plan{Comparison with 2-D image modelling method.}
        \item \plan{``Correct'' the PSF smeearing effect at the center with the help of 
            residual correct 2-D model.}
    \end{enumerate}

%% ------------------------------------------------------------------------------------ %% 

\section{Summary}

   


%% ------------------------------------------------------------------------------------ %% 
  
\acknowledgements
  
  The Hyper Suprime-Cam (HSC) collaboration includes the astronomical
  communities of Japan and Taiwan, and Princeton University.  The HSC
  instrumentation and software were developed by the National
  Astronomical Observatory of Japan (NAOJ), the Kavli Institute for the
  Physics and Mathematics of the Universe (Kavli IPMU), the University
  of Tokyo, the High Energy Accelerator Research Organization (KEK), the
  Academia Sinica Institute for Astronomy and Astrophysics in Taiwan
  (ASIAA), and Princeton University.  Funding was contributed by the FIRST 
  program from Japanese Cabinet Office, the Ministry of Education, Culture, 
  Sports, Science and Technology (MEXT), the Japan Society for the 
  Promotion of Science (JSPS),  Japan Science and Technology Agency 
  (JST),  the Toray Science  Foundation, NAOJ, Kavli IPMU, KEK, ASIAA,  
  and Princeton University.
   
  Funding for SDSS-III has been provided by the Alfred P. Sloan Foundation, the
  Participating Institutions, the National Science Foundation, and the U.S.
  Department of Energy. The SDSS-III web site is http://www.sdss3.org.  SDSS-III
  is managed by the Astrophysical Research Consortium for the Participating
  Institutions of the SDSS-III Collaboration including the University of
  Arizona, the Brazilian Participation Group, Brookhaven National Laboratory,
  University of Cambridge, University of Florida, the French Participation
  Group, the German Participation Group, the Instituto de Astrofisica de
  Canarias, the Michigan State/Notre Dame/JINA Participation Group, Johns
  Hopkins University, Lawrence Berkeley National Laboratory, Max Planck
  Institute for Astrophysics, New Mexico State University, New York University,
  Ohio State University, Pennsylvania State University, University of
  Portsmouth, Princeton University, the Spanish Participation Group, University
  of Tokyo, University of Utah, Vanderbilt University, University of Virginia,
  University of Washington, and Yale University.
  
  \todo{Full acknowledgement}\\
  \begin{itemize}
      \item Acknowledgements for Kevin and Alexie's funding 
      \item Acknowledgements for the Python libraries
  \end{itemize}

%%%%%%%%%%: Bibligraphic Section %%%%%%%%%%

%\begin{thebibliography}{}

  %\bibitem[Arnold et al.(2011)]{2011ApJ...736L..26A} Arnold, J.~A., Romanowsky,
    %A.~J., Brodie, J.~P., et al.\ 2011, \apjl, 736, LL26 
    
%\end{thebibliography}{}

\bibliography{hsc_massive.bbl}
\bibliographystyle{apj}

%%%%%%%%%%: Appendix Section %%%%%%%%%%%%

\appendix

\section{A. Extraction of 1-D surface brightness profile} 

\section{B. Derive average mass-to-light ratio using \texttt{iSEDFit}} 

%%%%%%%%%%%%: Figures Section %%%%%%%%%%%%

% Fig. 1
\clearpage
\figurenum{1}
\begin{figure}
    \centering 
    \includegraphics[width=15.5cm]{fig/fig1.png}
    \caption{Figure.1\todo{Caption}}
    \label{figure:1}
\end{figure}

% Fig. 2
\clearpage
\figurenum{2}
\begin{figure}
    \centering 
    \includegraphics[width=13.5cm]{fig/massive_s15b_g15_sample.png}
    \caption{Figure.2\todo{Caption}}
    \label{figure:2}
\end{figure}

% Fig. 3
\clearpage
\figurenum{3}
\begin{figure}
    \centering 
    \includegraphics[width=13.5cm]{fig/massive_s15b_zuse_hist1.png}
    \caption{Figure.3\todo{Caption}}
    \label{figure:3}
\end{figure}

% Fig. 4
\clearpage
\figurenum{4}
\begin{figure}
    \centering 
    \includegraphics[width=15.5cm]{fig/redbcg_1_HSC-I_full_imgsub_ellip_default_sum.png}
    \caption{Figure.4\todo{Caption}}
    \label{figure:4}
\end{figure}

% Fig. 5
\clearpage
\figurenum{5}
\begin{figure}
    \centering 
    \includegraphics[width=14.0cm]{fig/redbcg_1_HSC-I_full_imgsub_ellip_default_compare.png}
    \caption{Figure.5\todo{Caption}}
    \label{figure:5}
\end{figure}

%%%%%%%%%%%%: End of the File %%%%%%%%%%%%

\end{CJK*}

\clearpage 

%%%%%%%%%%%: Possible Tables %%%%%%%%%%%%%
%\begin{deluxetable}{c ccc ccc}[b!]
\tabletypesize{\scriptsize}
\tablewidth{0pt}
\tablecolumns{7}
\tablenum{1}
\tablecaption{Median \mden{} Profiles of Massive Galaxies in Different Stellar Mass Bins}
%% ------------------------------------------------------------------------------------ %% 
\tablehead{
    \colhead{$R/\mathrm{kpc}$} & 
    \multicolumn{3}{c}{\mden{} ($\log (M_{\odot}/\mathrm{kpc}^2)$) for \rbcg{}} &
    \multicolumn{3}{c}{\mden{} ($\log (M_{\odot}/\mathrm{kpc}^2)$) for \nbcg{}}
	\vspace{1.4ex}
    %------------------------------------------------------------------------------------%
    \nl 
    \colhead{} & 
    \colhead{$\log \frac{M_{\star,100\mathrm{kpc}}}{M_{\odot}}\in$[11.5, 11.7]} & 
    \colhead{[11.7, 11.9]} & 
    \colhead{[11.9, 12.2]}\hspace{2.0ex} & 
    \colhead{[11.2, 11.5]} & 
    \colhead{[11.5, 11.7]} & 
    \colhead{[11.7, 11.9]}
    %------------------------------------------------------------------------------------%
	\vspace{1.6ex}
    %------------------------------------------------------------------------------------%
    \nl
    \colhead{    (1)} &
    \colhead{    (2)} &
    \colhead{    (3)} &
    \colhead{    (4)} &
    \colhead{    (5)} &
    \colhead{    (6)} &
    \colhead{    (7)}
    %------------------------------------------------------------------------------------%
}
%% ------------------------------------------------------------------------------------ %% 
\startdata
%% ------------------------------------------------------------------------------------ %% 

 0.0 & $ 9.30\substack{+0.02 \\ -0.02}$ &$ 9.33\substack{+0.02 \\ -0.02}$ &$ 9.34\substack{+0.02 \\ -0.03}$ &$ 9.36\substack{+0.00 \\ -0.00}$ &$ 9.27\substack{+0.00 \\ -0.00}$ &$ 9.32\substack{+0.01 \\ -0.01}$ \\
 0.5 & $ 9.28\substack{+0.02 \\ -0.02}$ &$ 9.31\substack{+0.02 \\ -0.02}$ &$ 9.32\substack{+0.03 \\ -0.02}$ &$ 9.33\substack{+0.00 \\ -0.00}$ &$ 9.25\substack{+0.00 \\ -0.00}$ &$ 9.30\substack{+0.01 \\ -0.01}$ \\
 0.8 & $ 9.25\substack{+0.02 \\ -0.02}$ &$ 9.28\substack{+0.02 \\ -0.02}$ &$ 9.30\substack{+0.02 \\ -0.02}$ &$ 9.29\substack{+0.00 \\ -0.00}$ &$ 9.22\substack{+0.00 \\ -0.00}$ &$ 9.27\substack{+0.01 \\ -0.01}$ \\
 1.2 & $ 9.22\substack{+0.02 \\ -0.02}$ &$ 9.25\substack{+0.02 \\ -0.02}$ &$ 9.28\substack{+0.02 \\ -0.02}$ &$ 9.24\substack{+0.00 \\ -0.00}$ &$ 9.19\substack{+0.00 \\ -0.00}$ &$ 9.24\substack{+0.01 \\ -0.01}$ \\
 1.5 & $ 9.16\substack{+0.02 \\ -0.01}$ &$ 9.20\substack{+0.02 \\ -0.02}$ &$ 9.24\substack{+0.02 \\ -0.02}$ &$ 9.17\substack{+0.00 \\ -0.00}$ &$ 9.14\substack{+0.00 \\ -0.00}$ &$ 9.20\substack{+0.01 \\ -0.01}$ \\
 1.9 & $ 9.10\substack{+0.01 \\ -0.01}$ &$ 9.14\substack{+0.02 \\ -0.01}$ &$ 9.19\substack{+0.02 \\ -0.02}$ &$ 9.08\substack{+0.00 \\ -0.00}$ &$ 9.08\substack{+0.00 \\ -0.00}$ &$ 9.15\substack{+0.01 \\ -0.01}$ \\
 2.2 & $ 9.03\substack{+0.01 \\ -0.01}$ &$ 9.08\substack{+0.01 \\ -0.01}$ &$ 9.14\substack{+0.02 \\ -0.02}$ &$ 9.00\substack{+0.00 \\ -0.00}$ &$ 9.01\substack{+0.00 \\ -0.00}$ &$ 9.09\substack{+0.01 \\ -0.01}$ \\
 2.5 & $ 8.97\substack{+0.01 \\ -0.01}$ &$ 9.03\substack{+0.01 \\ -0.01}$ &$ 9.10\substack{+0.02 \\ -0.02}$ &$ 8.93\substack{+0.00 \\ -0.00}$ &$ 8.96\substack{+0.00 \\ -0.00}$ &$ 9.04\substack{+0.01 \\ -0.01}$ \\
 2.9 & $ 8.90\substack{+0.01 \\ -0.01}$ &$ 8.97\substack{+0.01 \\ -0.01}$ &$ 9.04\substack{+0.02 \\ -0.02}$ &$ 8.85\substack{+0.00 \\ -0.00}$ &$ 8.89\substack{+0.00 \\ -0.00}$ &$ 8.98\substack{+0.01 \\ -0.01}$ \\
 3.2 & $ 8.83\substack{+0.01 \\ -0.01}$ &$ 8.90\substack{+0.01 \\ -0.01}$ &$ 8.99\substack{+0.01 \\ -0.02}$ &$ 8.77\substack{+0.00 \\ -0.00}$ &$ 8.82\substack{+0.00 \\ -0.00}$ &$ 8.91\substack{+0.00 \\ -0.01}$ \\
 3.6 & $ 8.75\substack{+0.01 \\ -0.01}$ &$ 8.83\substack{+0.01 \\ -0.01}$ &$ 8.93\substack{+0.01 \\ -0.01}$ &$ 8.68\substack{+0.00 \\ -0.00}$ &$ 8.74\substack{+0.00 \\ -0.00}$ &$ 8.84\substack{+0.01 \\ -0.01}$ \\
 4.0 & $ 8.69\substack{+0.01 \\ -0.01}$ &$ 8.78\substack{+0.01 \\ -0.01}$ &$ 8.88\substack{+0.01 \\ -0.01}$ &$ 8.61\substack{+0.00 \\ -0.00}$ &$ 8.69\substack{+0.00 \\ -0.00}$ &$ 8.79\substack{+0.01 \\ -0.01}$ \\
 4.3 & $ 8.63\substack{+0.01 \\ -0.01}$ &$ 8.72\substack{+0.01 \\ -0.01}$ &$ 8.83\substack{+0.01 \\ -0.01}$ &$ 8.55\substack{+0.00 \\ -0.00}$ &$ 8.63\substack{+0.00 \\ -0.00}$ &$ 8.73\substack{+0.01 \\ -0.01}$ \\
 4.7 & $ 8.56\substack{+0.01 \\ -0.01}$ &$ 8.67\substack{+0.01 \\ -0.01}$ &$ 8.78\substack{+0.01 \\ -0.01}$ &$ 8.48\substack{+0.00 \\ -0.00}$ &$ 8.56\substack{+0.00 \\ -0.00}$ &$ 8.68\substack{+0.01 \\ -0.00}$ \\
 6.1 & $ 8.36\substack{+0.01 \\ -0.01}$ &$ 8.48\substack{+0.01 \\ -0.01}$ &$ 8.61\substack{+0.01 \\ -0.01}$ &$ 8.26\substack{+0.00 \\ -0.00}$ &$ 8.36\substack{+0.00 \\ -0.00}$ &$ 8.48\substack{+0.01 \\ -0.00}$ \\
 7.4 & $ 8.20\substack{+0.01 \\ -0.01}$ &$ 8.34\substack{+0.01 \\ -0.01}$ &$ 8.47\substack{+0.01 \\ -0.02}$ &$ 8.09\substack{+0.00 \\ -0.00}$ &$ 8.19\substack{+0.00 \\ -0.00}$ &$ 8.33\substack{+0.01 \\ -0.01}$ \\
 8.8 & $ 8.07\substack{+0.01 \\ -0.01}$ &$ 8.22\substack{+0.01 \\ -0.01}$ &$ 8.36\substack{+0.02 \\ -0.02}$ &$ 7.94\substack{+0.00 \\ -0.00}$ &$ 8.05\substack{+0.00 \\ -0.00}$ &$ 8.20\substack{+0.01 \\ -0.01}$ \\
10.3 & $ 7.94\substack{+0.01 \\ -0.01}$ &$ 8.10\substack{+0.01 \\ -0.01}$ &$ 8.25\substack{+0.02 \\ -0.02}$ &$ 7.79\substack{+0.00 \\ -0.00}$ &$ 7.91\substack{+0.00 \\ -0.00}$ &$ 8.07\substack{+0.00 \\ -0.01}$ \\
11.7 & $ 7.82\substack{+0.01 \\ -0.01}$ &$ 8.00\substack{+0.01 \\ -0.01}$ &$ 8.16\substack{+0.02 \\ -0.02}$ &$ 7.66\substack{+0.00 \\ -0.00}$ &$ 7.79\substack{+0.00 \\ -0.00}$ &$ 7.96\substack{+0.01 \\ -0.01}$ \\
13.0 & $ 7.73\substack{+0.01 \\ -0.01}$ &$ 7.92\substack{+0.01 \\ -0.01}$ &$ 8.08\substack{+0.02 \\ -0.02}$ &$ 7.55\substack{+0.00 \\ -0.00}$ &$ 7.70\substack{+0.00 \\ -0.00}$ &$ 7.87\substack{+0.01 \\ -0.01}$ \\
14.5 & $ 7.64\substack{+0.01 \\ -0.01}$ &$ 7.84\substack{+0.02 \\ -0.01}$ &$ 8.01\substack{+0.02 \\ -0.02}$ &$ 7.45\substack{+0.00 \\ -0.00}$ &$ 7.60\substack{+0.00 \\ -0.00}$ &$ 7.78\substack{+0.01 \\ -0.01}$ \\
16.0 & $ 7.55\substack{+0.02 \\ -0.02}$ &$ 7.75\substack{+0.01 \\ -0.01}$ &$ 7.94\substack{+0.02 \\ -0.02}$ &$ 7.34\substack{+0.00 \\ -0.00}$ &$ 7.50\substack{+0.00 \\ -0.00}$ &$ 7.70\substack{+0.01 \\ -0.01}$ \\
17.3 & $ 7.48\substack{+0.02 \\ -0.02}$ &$ 7.69\substack{+0.01 \\ -0.01}$ &$ 7.88\substack{+0.02 \\ -0.02}$ &$ 7.25\substack{+0.00 \\ -0.00}$ &$ 7.42\substack{+0.00 \\ -0.00}$ &$ 7.63\substack{+0.01 \\ -0.01}$ \\
18.7 & $ 7.41\substack{+0.02 \\ -0.02}$ &$ 7.62\substack{+0.02 \\ -0.02}$ &$ 7.83\substack{+0.02 \\ -0.02}$ &$ 7.16\substack{+0.00 \\ -0.00}$ &$ 7.34\substack{+0.00 \\ -0.00}$ &$ 7.56\substack{+0.01 \\ -0.01}$ \\
22.6 & $ 7.23\substack{+0.02 \\ -0.02}$ &$ 7.47\substack{+0.02 \\ -0.02}$ &$ 7.69\substack{+0.02 \\ -0.02}$ &$ 6.94\substack{+0.00 \\ -0.00}$ &$ 7.15\substack{+0.00 \\ -0.00}$ &$ 7.38\substack{+0.01 \\ -0.01}$ \\
26.1 & $ 7.10\substack{+0.02 \\ -0.02}$ &$ 7.34\substack{+0.02 \\ -0.02}$ &$ 7.58\substack{+0.02 \\ -0.02}$ &$ 6.76\substack{+0.00 \\ -0.00}$ &$ 7.00\substack{+0.00 \\ -0.00}$ &$ 7.25\substack{+0.01 \\ -0.01}$ \\
30.0 & $ 6.96\substack{+0.02 \\ -0.02}$ &$ 7.22\substack{+0.02 \\ -0.02}$ &$ 7.47\substack{+0.02 \\ -0.02}$ &$ 6.58\substack{+0.00 \\ -0.00}$ &$ 6.84\substack{+0.00 \\ -0.00}$ &$ 7.11\substack{+0.01 \\ -0.01}$ \\
33.7 & $ 6.84\substack{+0.02 \\ -0.02}$ &$ 7.11\substack{+0.02 \\ -0.02}$ &$ 7.37\substack{+0.02 \\ -0.02}$ &$ 6.42\substack{+0.01 \\ -0.01}$ &$ 6.71\substack{+0.00 \\ -0.00}$ &$ 6.99\substack{+0.01 \\ -0.01}$ \\
37.8 & $ 6.72\substack{+0.02 \\ -0.02}$ &$ 7.01\substack{+0.02 \\ -0.02}$ &$ 7.27\substack{+0.02 \\ -0.02}$ &$ 6.27\substack{+0.01 \\ -0.01}$ &$ 6.57\substack{+0.00 \\ -0.00}$ &$ 6.87\substack{+0.01 \\ -0.01}$ \\
41.6 & $ 6.62\substack{+0.02 \\ -0.02}$ &$ 6.92\substack{+0.02 \\ -0.02}$ &$ 7.18\substack{+0.02 \\ -0.02}$ &$ 6.14\substack{+0.01 \\ -0.01}$ &$ 6.46\substack{+0.00 \\ -0.01}$ &$ 6.77\substack{+0.01 \\ -0.01}$ \\
45.7 & $ 6.50\substack{+0.03 \\ -0.03}$ &$ 6.82\substack{+0.02 \\ -0.02}$ &$ 7.09\substack{+0.02 \\ -0.02}$ &$ 6.01\substack{+0.01 \\ -0.01}$ &$ 6.34\substack{+0.01 \\ -0.00}$ &$ 6.66\substack{+0.01 \\ -0.01}$ \\
49.3 & $ 6.42\substack{+0.03 \\ -0.03}$ &$ 6.74\substack{+0.03 \\ -0.02}$ &$ 7.02\substack{+0.02 \\ -0.02}$ &$ 5.91\substack{+0.01 \\ -0.01}$ &$ 6.25\substack{+0.01 \\ -0.01}$ &$ 6.57\substack{+0.01 \\ -0.01}$ \\
53.1 & $ 6.34\substack{+0.03 \\ -0.03}$ &$ 6.67\substack{+0.03 \\ -0.02}$ &$ 6.94\substack{+0.02 \\ -0.02}$ &$ 5.81\substack{+0.01 \\ -0.01}$ &$ 6.15\substack{+0.01 \\ -0.01}$ &$ 6.48\substack{+0.01 \\ -0.01}$ \\
57.2 & $ 6.25\substack{+0.03 \\ -0.03}$ &$ 6.59\substack{+0.03 \\ -0.03}$ &$ 6.86\substack{+0.03 \\ -0.03}$ &$ 5.71\substack{+0.01 \\ -0.01}$ &$ 6.06\substack{+0.01 \\ -0.01}$ &$ 6.40\substack{+0.01 \\ -0.01}$ \\
61.5 & $ 6.18\substack{+0.03 \\ -0.03}$ &$ 6.52\substack{+0.03 \\ -0.03}$ &$ 6.78\substack{+0.03 \\ -0.03}$ &$ 5.62\substack{+0.01 \\ -0.01}$ &$ 5.96\substack{+0.01 \\ -0.01}$ &$ 6.31\substack{+0.01 \\ -0.01}$ \\
65.1 & $ 6.11\substack{+0.04 \\ -0.04}$ &$ 6.46\substack{+0.03 \\ -0.03}$ &$ 6.72\substack{+0.03 \\ -0.03}$ &$ 5.55\substack{+0.01 \\ -0.01}$ &$ 5.89\substack{+0.01 \\ -0.01}$ &$ 6.23\substack{+0.01 \\ -0.01}$ \\
68.8 & $ 6.02\substack{+0.04 \\ -0.04}$ &$ 6.39\substack{+0.03 \\ -0.03}$ &$ 6.66\substack{+0.03 \\ -0.03}$ &$ 5.48\substack{+0.01 \\ -0.01}$ &$ 5.82\substack{+0.01 \\ -0.01}$ &$ 6.16\substack{+0.01 \\ -0.01}$ \\
73.7 & $ 5.95\substack{+0.04 \\ -0.05}$ &$ 6.32\substack{+0.03 \\ -0.03}$ &$ 6.58\substack{+0.03 \\ -0.04}$ &$ 5.40\substack{+0.01 \\ -0.01}$ &$ 5.73\substack{+0.01 \\ -0.01}$ &$ 6.07\substack{+0.01 \\ -0.01}$ \\
78.9 & $ 5.88\substack{+0.05 \\ -0.05}$ &$ 6.24\substack{+0.03 \\ -0.03}$ &$ 6.50\substack{+0.03 \\ -0.04}$ &$ 5.33\substack{+0.01 \\ -0.01}$ &$ 5.65\substack{+0.01 \\ -0.01}$ &$ 5.99\substack{+0.01 \\ -0.01}$ \\
84.3 & $ 5.79\substack{+0.06 \\ -0.06}$ &$ 6.17\substack{+0.03 \\ -0.03}$ &$ 6.42\substack{+0.04 \\ -0.04}$ &$ 5.26\substack{+0.01 \\ -0.01}$ &$ 5.58\substack{+0.01 \\ -0.01}$ &$ 5.91\substack{+0.02 \\ -0.02}$ \\
91.2 & $ 5.71\substack{+0.05 \\ -0.06}$ &$ 6.08\substack{+0.03 \\ -0.03}$ &$ 6.31\substack{+0.05 \\ -0.05}$ &$ 5.19\substack{+0.01 \\ -0.01}$ &$ 5.48\substack{+0.01 \\ -0.01}$ &$ 5.81\substack{+0.02 \\ -0.02}$ \\
98.5 & $ 5.65\substack{+0.05 \\ -0.05}$ &$ 6.00\substack{+0.03 \\ -0.03}$ &$ 6.22\substack{+0.06 \\ -0.07}$ &$ 5.12\substack{+0.01 \\ -0.01}$ &$ 5.40\substack{+0.01 \\ -0.01}$ &$ 5.69\substack{+0.02 \\ -0.02}$ \\
106.2 & $ 5.55\substack{+0.06 \\ -0.06}$ &$ 5.92\substack{+0.03 \\ -0.03}$ &$ 6.15\substack{+0.05 \\ -0.07}$ &$ 5.05\substack{+0.01 \\ -0.01}$ &$ 5.30\substack{+0.01 \\ -0.01}$ &$ 5.60\substack{+0.02 \\ -0.02}$ \\
115.7 & $ 5.46\substack{+0.06 \\ -0.06}$ &$ 5.81\substack{+0.03 \\ -0.03}$ &$ 6.08\substack{+0.04 \\ -0.05}$ &$ 4.98\substack{+0.01 \\ -0.02}$ &$ 5.21\substack{+0.01 \\ -0.01}$ &$ 5.51\substack{+0.02 \\ -0.02}$ \\
123.0 & $ 5.45\substack{+0.05 \\ -0.06}$ &$ 5.72\substack{+0.04 \\ -0.04}$ &$ 6.00\substack{+0.05 \\ -0.04}$ &$ 4.91\substack{+0.02 \\ -0.01}$ &$ 5.16\substack{+0.01 \\ -0.01}$ &$ 5.44\substack{+0.02 \\ -0.02}$ \\
129.0 & $ 5.43\substack{+0.05 \\ -0.05}$ &$ 5.67\substack{+0.04 \\ -0.04}$ &$ 5.93\substack{+0.05 \\ -0.05}$ &$ 4.85\substack{+0.02 \\ -0.02}$ &$ 5.11\substack{+0.01 \\ -0.01}$ &$ 5.37\substack{+0.02 \\ -0.02}$ \\
136.8 & $ 5.37\substack{+0.06 \\ -0.06}$ &$ 5.63\substack{+0.04 \\ -0.04}$ &$ 5.82\substack{+0.06 \\ -0.06}$ &$ 4.78\substack{+0.02 \\ -0.02}$ &$ 5.03\substack{+0.02 \\ -0.02}$ &$ 5.29\substack{+0.02 \\ -0.02}$ \\
145.0 & $ 5.29\substack{+0.06 \\ -0.08}$ &$ 5.57\substack{+0.04 \\ -0.04}$ &$ 5.72\substack{+0.08 \\ -0.08}$ &$ 4.71\substack{+0.02 \\ -0.02}$ &$ 4.96\substack{+0.02 \\ -0.02}$ &$ 5.21\substack{+0.02 \\ -0.03}$
%%------------------------------------------------------------------------------------ %% 
\enddata
%% ------------------------------------------------------------------------------------ %% 
\tablecomments{
    Median \mden{} profiles of massive \rbcg{} and \nbcg{} galaxies in different \mstar{} 
    bins.  
    Col.~(1) Radius in kpc.
    Col.~(2) Median \mden{} profile for \rbcg{} galaxies with 
        $11.5 \leq$\logmtot$< 11.7$. 
    Col.~(3) Median \mden{} profile for \rbcg{} galaxies with 
        $11.7 \leq$\logmtot$< 11.9$. 
    Col.~(4) Median \mden{} profile for \rbcg{} galaxies with 
        $11.9 \leq$\logmtot$< 12.1$. 
    Col.~(5) Median \mden{} profile for \nbcg{} galaxies with 
        $11.2 \leq$\logmtot$< 11.5$. 
    Col.~(6) Median \mden{} profile for \nbcg{} galaxies with 
        $11.5 \leq$\logmtot$< 11.7$. 
    Col.~(7) Median \mden{} profile for \nbcg{} galaxies with 
        $11.7 \leq$\logmtot$< 11.9$. 
    The full table is available in electronic version.
}

\end{deluxetable}


\end{document}